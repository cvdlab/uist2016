\documentclass{sigchi}

% Use this command to override the default ACM copyright statement (e.g. for preprints).
% Consult the conference website for the camera-ready copyright statement.


%% EXAMPLE BEGIN -- HOW TO OVERRIDE THE DEFAULT COPYRIGHT STRIP -- (July 22, 2013 - Paul Baumann)
% \toappear{Permission to make digital or hard copies of all or part of this work for personal or classroom use is 	granted without fee provided that copies are not made or distributed for profit or commercial advantage and that copies bear this notice and the full citation on the first page. Copyrights for components of this work owned by others than ACM must be honored. Abstracting with credit is permitted. To copy otherwise, or republish, to post on servers or to redistribute to lists, requires prior specific permission and/or a fee. Request permissions from permissions@acm.org. \\
% {\emph{CHI'14}}, April 26--May 1, 2014, Toronto, Canada. \\
% Copyright \copyright~2014 ACM ISBN/14/04...\$15.00. \\
% DOI string from ACM form confirmation}
%% EXAMPLE END -- HOW TO OVERRIDE THE DEFAULT COPYRIGHT STRIP -- (July 22, 2013 - Paul Baumann)


% Arabic page numbers for submission.
% Remove this line to eliminate page numbers for the camera ready copy
% \pagenumbering{arabic}


% Load basic packages
%\usepackage[latin1]{inputenc} % Windows
\usepackage[utf8x]{inputenc} % Linux (unicode package needed)
% \usepackage[applemac]{inputenc} % Mac
\usepackage{balance}  % to better equalize the last page
\usepackage{graphics} % for EPS, load graphicx instead
\usepackage{times}    % comment if you want LaTeX's default font
\usepackage{url}      % llt: nicely formatted URLs

% llt: Define a global style for URLs, rather that the default one
\makeatletter
\def\url@leostyle{%
  \@ifundefined{selectfont}{\def\UrlFont{\sf}}{\def\UrlFont{\small\bf\ttfamily}}}
\makeatother
\urlstyle{leo}


% To make various LaTeX processors do the right thing with page size.
\def\pprw{8.5in}
\def\pprh{11in}
\special{papersize=\pprw,\pprh}
\setlength{\paperwidth}{\pprw}
\setlength{\paperheight}{\pprh}
\setlength{\pdfpagewidth}{\pprw}
\setlength{\pdfpageheight}{\pprh}

% Make sure hyperref comes last of your loaded packages,
% to give it a fighting chance of not being over-written,
% since its job is to redefine many LaTeX commands.
\usepackage[pdftex]{hyperref}
\hypersetup{
pdftitle={SIGCHI Conference Proceedings Format},
pdfauthor={LaTeX},
pdfkeywords={SIGCHI, proceedings, archival format},
bookmarksnumbered,
pdfstartview={FitH},
colorlinks,
citecolor=black,
filecolor=black,
linkcolor=black,
urlcolor=black,
breaklinks=true,
}

% create a shortcut to typeset table headings
\newcommand\tabhead[1]{\small\textbf{#1}}


% End of preamble. Here it comes the document.
\begin{document}

\title{Quasi-Photorealistic Indoor Scene Creation, Editing and Navigation as a Service}

\numberofauthors{4}
\author{
\alignauthor Federico Spini\\
      \affaddr{Dipartimento di Ingegneria}\\
      \affaddr{Universit\`a Roma Tre}\\
      \affaddr{Rome, Italy}\\
      \email{spini@dia.uniroma3.it}
\alignauthor Enrico Marino\\
      \affaddr{Dipartimento di Ingegneria}\\
      \affaddr{Universit\`a Roma Tre}\\
      \affaddr{Rome, Italy}\\
      \email{marino@dia.uniroma3.it}
\alignauthor Michele D'Antmi\\
      \affaddr{Dipartimento di Ingegneria}\\
      \affaddr{Universit\`a Roma Tre}\\
      \affaddr{Rome, Italy}\\
      \email{dantimi@dia.uniroma3.it}
\alignauthor Edoardo Carra\\
      \affaddr{Dipartimento di Ingegneria}\\
      \affaddr{Universit\`a Roma Tre}\\
      \affaddr{Rome, Italy}\\
      \email{carra@dia.uniroma3.it}
}

\maketitle

\begin{abstract}
Lorem ipsum dolor sit amet, consectetur adipisicing elit, sed do eiusmod tempor incididunt ut labore et dolore magna aliqua. Ut enim ad minim veniam, quis nostrud exercitation ullamco laboris nisi ut aliquip ex ea commodo consequat. Duis aute irure dolor in reprehenderit in voluptate velit esse cillum dolore eu fugiat nulla pariatur. Excepteur sint occaecat cupidatat non proident, sunt in culpa qui officia deserunt mollit anim id est laborum.
\end{abstract}

\keywords{
	Guides; instructions; author's kit; conference publications;
	keywords should be separated by a semi-colon. \newline
	\textcolor{red}{Optional section to be included in your final version,
  but strongly encouraged.}
}

\category{H.5.m.}{Information Interfaces and Presentation (e.g. HCI)}{Miscellaneous}

See: \url{http://www.acm.org/about/class/1998/}
for more information and the full list of ACM classifiers
and descriptors. \newline
\textcolor{red}{Optional section to be included in your final version,
but strongly encouraged. On the submission page only the classifiers’
letter-number combination will need to be entered.}

\section{Observations}
\label{sec:Observations}

\subsection{Title} % (fold)
\label{sub:title}

The paper title should be shortened: valid alternatives are

\begin{itemize}
  \item \emph{``Quasi-Photorealistic Indoor Scene Managemenet as a Service''}
  \item \emph{``Quasi-Photorealistic Indoor Scene Lifecycle Management as a Service''}
  \item \emph{``Quasi-Photorealistic Indoor Scene Manupulation as a Service''}
  \item \emph{``Quasi-Photorealistic Indoor Scene Manupulation and Navigation as a Service''}
  \end{itemize}

% subsection title (end)

\subsection{Video} % (fold)
\label{sub:video}

For the video, we were thinking about the realization ($1/4$ of the time) and the navigation ($3/4$ of the time) of a small museum (4 to 6 rooms), in which are exposed and a few statues and some pictures (simple to realize and impressive thanks to the good translucent effect of the frame glass, each one illuminated by a direct spotlight).

% subsection video (end)


\section{Introduction}  % (fold)
\label{sec:introduction}

Si deve iniziare contestualizzando le {\bf applicazioni} dei risultati anche futuri e futuribili di una tecnologia di questo tipo, il cardine della tecnologia è il seguente: si mette in condizione l'utente che abbia un minimo di competenze di grafica 3D ed un computer con accesso ad internet di 1) mettere a disposizione di altri tramite un indirizo web una scena 3D di interno navigabile con qualità quasi-fotorealistica, 2) avendo realizzato tale scena in autonomia sempre collegandosi ad un sito internet.


{\em Introductory argumentations}: 1) user-centric 3D application; 2)self-build 3D application.



{\em Enabled practical applications}: 1) Collaborative architecture for the end user; 2) virtual web-accessible self-builded visits of indoor spaces (inaccessible for any reason).



{\em Enabling social factors}. Native FPS gamers feel confident moving in a virtual environment in which instead of shoot zombie or aliens, they can take a look for example of the architectural project of their renewed home.


3D graphics techniques and algorithms are well studied and exploited right now. 

The introduction of WebGL offers the possibility to bring to the people via the Web medium brand new and useful experiences.

But the web is accessible form a variety of different devices. Most of them are provided with a limited hardware.



The main contribution of this work is the definition and implementation of a workflow completely based on the Web infrastructure, to let the user create, refine, edit and navigate an indoor 3D scene with a quasi-photorealistic quality.


We hypothesized a system in which any user with a very basic graphic knowhow is put in a position in which, by connecting to a website, he is autonomously able to build a scene, require remote processing and finally make use of the resulting quasi-photorealistic web navigable 3D scene.


Abbiamo ipotizzato un sistema con il quale ogni utente con un minimo livello di conoscenze di grafica 3D, può mettere a disposizione del mondo attraverso un sito web una versione navigabile e quasi-fotorealistica di un ambiente interno da lui stesso autonomamente realizzato, sempre cnnettendosi ad un sito web, quindi senza necessità di installare (ed eventualmente ancor prima di acquistare) un prodotto software.




% section introduction (end)








\section{User Experience} % (fold) #
\label{sec:user_experience}

\subsection{Scene creation} % (fold)
\label{sub:scene_creation}

% subsection scene_creation (end)

\subsection{Rapid feedback gatering} % (fold)
\label{sub:rapid_feedback}

% subsection rapid_feedback (end)
% section user_user_experience (end)








\section{Rendering Model} % (fold)
\label{sec:rendering_model}


\subsection{Quasi-photorealistic techiniques}
\label{sub:quasi_photorealistic_techniques}

They are:

\begin{itemize}
  \item Precise physical based shadow rendering (INHERIT THE MOST POSSIBLE FROM BLENDER CYCLE RENDERING)
  \item Refraction map
  \item Reflection map
  \item Skybox
  \item ...
\end{itemize}

% subsection quasi_photorealistic_techniques (end)




\subsection{Enabling Conditions} % (fold)
\label{sub:enabling_conditions}

This section should report about enabling conditions for lightweight render loop, i.e. indoor still scene, for which shadows can be precomputed.

% subsection enabling_conditions (end)

\subsection{Lightweight Render Loop} % (fold)
\label{sub:lightweight_render_loop}

It is the case to merely list the numerous precautions and devices enabling such a simplified render loop.

% subsection lightweight_render_loop (end)

% section rendering_model (end)
















\section{Architecture} % (fold)
\label{sec:architecture}

In this section will be introduced the three constitutive components of the systems:

\begin{enumerate}
  \item the {\tt Editor}
  \item the {\tt Backing service}
  \item the {\tt Navigator}
\end{enumerate}

% section architecture (end)



\subsection{Entity mapping} % (fold)
\label{sub:entity_mapping}

Here goes the pseudo-code of the mapping algorithm.

% subsection entity_mapping (end)



\section{Experimental results} % (fold)
\label{sec:experimental_results}

Experimental data are the salt of the research. It is probably the case to add some data about user interaction, being the paper mainly focused on that aspect.

% section experimental_results (end)


\section{Discussion}  % (fold)
\label{sec:discussion}

This section is devoted to the comparison with other systems

\begin{enumerate}
  \item \em{Ikea FloorPlanner}
  \item \em{http://www.homestyler.com/floorplan/}
  \item \em{https://www.shapespark.com}
  \item \em{http://cgcloud.pro}
  \item \em{http://www.sketchup.com/}
\end{enumerate}

% section discussion (end)


\section{Conclusion}  % (fold)
\label{sec:conclusion}

Recap of the whole work

% section conclusion (end)


\section{Acknowledgments}  % (fold)
\label{sec:acknowledgments}

I would insert a sincerely thanks:

\begin{itemize}
  \item professor Alberto Paoluzzi
  \item Christian Vadalà
\end{itemize}

% section acknowledgments (end)

% Balancing columns in a ref list is a bit of a pain because you
% either use a hack like flushend or balance, or manually insert
% a column break.  http://www.tex.ac.uk/cgi-bin/texfaq2html?label=balance
% multicols doesn't work because we're already in two-column mode,
% and flushend isn't awesome, so I choose balance.  See this
% for more info: http://cs.brown.edu/system/software/latex/doc/balance.pdf
%
% Note that in a perfect world balance wants to be in the first
% column of the last page.
%
% If balance doesn't work for you, you can remove that and
% hard-code a column break into the bbl file right before you
% submit:
%
% http://stackoverflow.com/questions/2149854/how-to-manually-equalize-columns-
% in-an-ieee-paper-if-using-bibtex
%
% Or, just remove \balance and give up on balancing the last page.
%
\balance

\bibliographystyle{acm-sigchi}
\bibliography{references}
\end{document}
